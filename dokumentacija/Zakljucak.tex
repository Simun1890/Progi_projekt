\chapter{Zaključak i budući rad}
		
		\textbf{\textit{dio 2. revizije}}\\

			Zadatak našeg tima bio je razvoj web aplikacije koja olakšava komunikaciju, rezervaciju termina pregleda, te prosljeđivanje uputnica i ispričnica između doktora, roditelja i škola. Nakon nešto manje od četiri mjeseca razvoja, ovo je večinski ostvareno kroz dvije faze razvoja.
			
			Prva faza započela je okupljanjem članova tima i dodjelom projektnog zadatka. Tijekom prve faze započeli smo fokusom na konceptualnu razradu implementacije aplikacije - njenog korisničkog sučelja, baze podataka, te omogučenih funkcionalnosti svakog tipa korsinika.
			Slijedila je edukacija svih članova tima o tehnologijama Springboot i React putem prisutstvovanja na stručnim predavanjima CROZ-a, te samostalnim radom.
			Nakon razrade koncepcije aplikacije i upoznavanja s tehnologijama krenuli smo u implementaciju sa fokusom na razvoj korisničkog sučelja aplikacije, te uspostave funkcionalne baze podataka dogovorene arhitekture. Kada je baza bila uspostavljena, backend tim je krenuo u početne faze povezivanja korisničkog sučelja na bazu, dok se frontend tim prebacio na dokumentaciju.
			Deployem aplikacije na servis "onrender" završila je prva faza.

			Druga faza je po pitanju kodiranja bila intenzivnija od prve. Zahvaljujući dobrim rješenjima korijenitih aspkeata aplikacije u prvoj fazi, druga je faza mogla biti puno više fokusirana na implementaciju traženih funkcionalnosti.
			Neovisan pristup razvoja frontenda i backenda uz naglasak na komunikaciju i držanje dogovorene arhitekture omogučio je brz razvoj, pri tom zadržavajući konzistentnost i lagano spajanje implementiranoga.
			Fokus druge faze bio je dovršetak korisničkog sučelja, te povezivanje istog na backend. Kasnije razdoblje faze fokusiralo se na pisanje dokumentaicje, testiranje i bugfixing.

			Od samog početka rada na zadatku, naš tim je naišao na svoj prvi izazov - nemogučnost sudjelovanja dva člana. Ovaj izazov je u dogovoru s CROZ-om i profesorima predmeta riješen dogovorenim izostavljanjem dva zahtjeva implementacije:
			\begin{itemize}
				\item Roditelji dobivaju obavijest/uputu od pedijatra ili liječnik obiteljske medicine nakon što je stigao nalaz iz laboratorija
				\item Liječnik obiteljske medicine ili pedijatar naručuje pacijenta na specijalistički pregled / postupak, a pacijent dobiva poruku s potvrdom o naručivanju s prikazom lokacija na mapi gdje taj pregled može obaviti s obzirom na mjesto stanovanja (OpenStreetMap)
			\end{itemize}
			Ovo smanjenje ljudstva je rezultiralo nešto sporijim razvojem i pisanjem dokumentacije, te potrebom članova da rade na više djelova projekta. Tako su članovi frontend podtima povremeno radili na backendu i obrnuto, te je cijeli tim bio zadužen za pisanje dokumentacije, umjesto jednog člana kojemu bi to bio primarni fokus.
			Unatoč navedenom, uspješno smo implementirani dogovorene zahtjeve u roku.
			Očekivali smo popriličan izazov u nepoznavanju tehnologija korištenih za razvoj aplikacije, no ispostavilo se kako su ih članovi tima brzo naučili i stvorili mogučnost razvoja u njima. Jedna dobra posljedica nedostatka članova u timu bila je i bolja upoznatost svih članova sa svim korištenim tehnologijama.
			
			Komunikacija između članova tima bila je ostvarena putem WhatsApp grupe što je omogučilo efektivnu i brzu razmjenu informacija.
			Ona se pokazala kao iznimno korisna, te je rezultirala dobrom standardizacijom koda i uspješnim držanjem dogovorenih normi i koncepata.
			Tim je cijelo razdoblje rada na aplikaciji proveo dobro informiran u vezi djelovanja ostalih članova tima.

			Naša preporuka za daljnji razvoj apliakcije bila bi implementacija mogučnosti koje smo mi morali izostaviti. Vjerujemo kako bi rezultirale kvalitetnijim korisničkim iskustvom.
			Također je moguć razvoj mobilne aplikacije koja bi mogla kvalitetnije implementirati notifikacije, te pružiti opećenito ugodnije korisničko iskustvo nego web preglednik.
			Za kraj, preporučili bi i razvoj jednostavnog chat-featurea za direktnu komunikaciju između doktora i roditelja.

			Rad na ovom projektu nam je dao nekoliko neprocijenjivih iskustava. Prije svega iskustvo timskog rada koje će svakome od nas biti neprocijenjivo u daljnjoj karijeri.
			Iskustvo "on-the-fly" učenja tehnologija je članovima pokazalo njihovu mogučnost brzog snalaženja. Naš početni izazov rezultirao je i sticanjem iskustva snalaženja u neočekivanim okolnostima, no još je bitnije unutar tima stvorio kulturu proaktivnog "uskakanja" u pomoć drugim članovima.
			Konačno, svi članovi tima su ovim projektom stekli znanje i iskustvo u često korištenim tehnologijama, te iskustvo potpunog razvoja projekta od svojeg početka do kraja.

		\eject 
